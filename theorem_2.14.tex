

\begin{suggestion}\label{2.14}
Пусть дан $M_n$ и точки $u_n(A), u_n(B), u_n(C), u_n(D)$ на его границе. Тогда при фиксированном $n$: $P(u_n(D)\leftrightarrow{}u_n(A))$ равняется вероятности протекания воды по шестиугольникам между ломаными $u_n(A)u_n(B)$ и $u_n(C)u_n(D)$, соответствующим отрезкам $AB$ и $CD$ при данном $n$.
\end{suggestion}

\begin{proof}
Для доказательства утверждения покажем равносильность наличия протекания между указанными ломанными и наличия красного пути между вершинами $u_n(A)$ и $u_n(D)$. Сразу заметим: способ покраски ломаной в красный цвет на границе выбран так, что можно считать внешний граничный слой клеток окрашенным в синий цвет на границе с ломаными $u_n(A)u_n(B)$ и $u_n(C)u_n(D)$, а на границе с ломаными $u_n(B)u_n(C)$ и $u_n(D)u_n(A)$. При этом клетки вне $M$, одними из середин ребер которых являются $u_n(A), u_n(B), u_n(C), u_n(D)$ окрашены в оба цвета одновременно.
\newline
$\Rightarrow{}$ Пусть имеется протекание между ломаными $u_n(A)u_n(B)$ и $u_n(D)u_n(C)$. Заметим, что данное протекание делит шестиугольники на две части. Тогда $u_n(A)$ не может быть соединена ни с $u_n(B)$, ни с $u_n(C)$, так как они лежат по разные стороны от протекания, а через него ребра не проходят. Значит $u_n(A)$ соединена с $u_n(D)$.
\newline
$\Leftarrow{}$ Пусть имеется красный путь из $u_n(A)$ в $u_n(D)$. Тогда ближайший слой клеток по разные стороны от этого пути - цепочки из одноцветных шестиугольников, одна из которых будет синей, а другая желтой. Рассмотрим синюю: некоторые ее шестиугольники может быть окажутся за границей $M$. Это возможно только на границах $u_n(A)u_n(B)$ и $u_n(C)u_n(D)$. Отбросим их, тогда получится, что у нас есть синий путь от ломаной $u_n(A)u_n(B)$ до $u_n(C)u_n(D)$, что и требовалось.

\end{proof}
\begin{suggestion} \label{3.1b}
Рассмотрим цепочку из $k$ шестиугольников из $M_n$, в которой соседние шестиугольники имеют общую сторону, и последний имеет общую сторону с первым. Обозначим через $w_1,\dots, w_{k}$ центры этих шестиугольников и положим $w_{k+1}:=w_1$. Тогда если $f(z)$ --- любая функция, удовлетворяющая тождеству треугольника и определенная на множестве середин сторон шестиугольников $n$-замощения из $M_n$, то
\begin{equation} \label{analit}
\sum\limits_{j=1}^{k} f\left(\frac{w_{j+1}+w_{j}}{2}\right) (w_{j+1}-w_{j})=0.
\end{equation}
\end{suggestion}


\begin{proof}[Доказательство предложения \ref{3.1b}] 
Соединим $w_j$ с $w_{j+1}$, получим цикл $\psi$. Рассмотрим все узлы шестиугольной решетки, которые расположены внутри цикла $\psi$. Напишем для каждого из этих узлов тождество треугольника. Сложим все полученные равенства. Тогда для любой середины $Z$ отрезка $AB$ находящейся строго внутри этого цикла, $f(z)$ будет посчитано с коэффициентами $z-a$ и $z-b$, то есть с коэффициентом $2z-a-b=0$. Тогда в сумме останутся только вершины вида $\frac{w_{j+1} + w_j}{2}$ с коэффициентами $\frac{w_{j+1} + w_j}{2} - w$, где $w$ - тот узел, в котором посчитана эта середина. Домножим полученное равенство на $2\sqrt{3}i$. Тогда вектор $\frac{w_{j+1} + w_j}{2} - w$ перейдет в вектор $w_{j+1} - w_j$. Мы получили равенство \ref{analit}.
\end{proof}
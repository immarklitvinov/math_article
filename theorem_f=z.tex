\begin{lemma}\label{f=z}

Тождественная функция $F(Z) = Z$ удовлетворяет предложениям: \ref{3.4a}, \ref{3.4b}, \ref{3.4c}.

\end{lemma}



\begin{proof}
По порядку докажем все три утверждения:

1. Тождественная функция непрерывна.

2. Покажем аналитичность:
\begin{multline*}
    \sum\limits_{j=1}^{l_n} F\left(\frac{w_{j+1,n}+w_{j,n}}{2}\right) (w_{j+1,n}-w_{j,n})= \sum\limits_{j=1}^{l_n} \frac{1}{2}(w_{j+1,n}+w_{j,n}) (w_{j+1,n}-w_{j,n})= \\ =  \frac{1}{2}\sum\limits_{j=1}^{l_n} (w_{j+1,n}^2-w_{j,n}^2)  = 0
\end{multline*}

Выражение обращается в нуль, поскольку каждое слагаемое в виде квадрата встречается с плюсом и минусом.

3. Граничное условие для тождественной функции верно в силу выбора расположения точек $A, B, C$. 
\end{proof}
\begin{suggestion} \label{3.4c}
 Для любой точки $D$ на границе треугольника $ABC$ функция $F(Z)$ из теоремы \ref{3.3} удовлетворяет условию: 
 $$
 F(D) \in 
 \begin{cases}
 [1; \tau{}^2] &\text{при D $\in$ [A, C],} \\
 [\tau{}^2; \tau{}] &\text{при D $\in$ [C, B],} \\
 [\tau{}; 1] &\text{при D $\in$ [B, A].} \\
 \end{cases}
 $$
 
\end{suggestion}



\begin{proof}
Пусть точка $D$ лежит на отрезке $[A, C]$. Тогда из определения функции $F(Z)$ имеем:
$$F(D) = \displaystyle \lim_{n\to\infty} {f_n(D_n)}$$
Поскольку $D$ лежит на отрезке $AC$, то ближайшая к ней середина $u_n(D)$ лежит на границе многоугольника $M_n$. Отсюда $P(u_n(B) \leftrightarrow u_n(D)) = 0$, так как если она не равна 0, то существует расположение ломаных, соединяющих $u_n(A)$ и $u_n(C)$, $u_n(B)$ и $u_n(D)$, но в таком случае они пересекаются $ \Rightarrow$ противоречие.
\\
Тогда: $\forall n \in \mathbb{N}: P(u_n(B) \leftrightarrow u_n(D) = 0 \Rightarrow f_n(u_n(Z))=P(u_n(D) \leftrightarrow u_n(A)) + {\tau{}^2}P(u_n(D) \leftrightarrow u_n(C))$
\\
Иными словами $f_n(u_n(Z))$ представляется в виде выпуклой комбинации векторов $1$ и $\tau{}^2$, и значит при всех $n$ лежит на отрезке $AC$. Тогда и предел, поскольку он существует, лежит на этом отрезке. Тогда $F(D) \in [AC]$. Что и требовалось. Остальные случаи разбираются аналогично.
\end{proof}